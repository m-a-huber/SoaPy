%------------------------------------------------------------------------------
% Beginning of journal.tex
%------------------------------------------------------------------------------
%
% AMS-LaTeX version 2 sample file for journals, based on amsart.cls.
%
%        ***     DO NOT USE THIS FILE AS A STARTER.      ***
%        ***  USE THE JOURNAL-SPECIFIC *.TEMPLATE FILE.  ***
%
% Replace amsart by the documentclass for the target journal, e.g., tran-l.
%
\documentclass[12pt]{amsart}

%     If your article includes graphics, uncomment this command.
\usepackage{graphicx}
\usepackage{cleveref}
\usepackage{color}
\usepackage{units}
\usepackage{enumitem}
\usepackage{tabularx}
  \newcolumntype{L}{>{\raggedright\arraybackslash}X}
\usepackage{ltablex}

\newtheorem{theorem}{Theorem}[section]
\newtheorem{lemma}[theorem]{Lemma}

\theoremstyle{definition}
\newtheorem{definition}{Definition}
\newtheorem{example}[theorem]{Example}
\newtheorem{xca}[theorem]{Exercise}

\theoremstyle{remark}
\newtheorem{remark}[theorem]{Remark}
\newtheorem*{remark*}{Remark}
\newtheorem*{example*}{Example}
\newtheorem*{examples*}{Examples}

\numberwithin{equation}{section}

%    Absolute value notation
\newcommand{\abs}[1]{\lvert#1\rvert}

%    Blank box placeholder for figures (to avoid requiring any
%    particular graphics capabilities for printing this document).
\newcommand{\blankbox}[2]{%
  \parbox{\columnwidth}{\centering
%    Set fboxsep to 0 so that the actual size of the box will match the
%    given measurements more closely.
    \setlength{\fboxsep}{0pt}%
    \fbox{\raisebox{0pt}[#2]{\hspace{#1}}}%
  }%
}

\setlength{\textwidth}{\paperwidth}
\addtolength{\textwidth}{-6cm}
\calclayout

\newcommand{\hfp}{\mathrm{HF}^{+}}
\newcommand{\Z}{\mathbb{Z}}

\begin{document}

\title{Instructions for SoaPy}
\vspace*{-3.5cm}
\maketitle
\thispagestyle{empty}

This package is intended to facilitate computations involving Seifert fibered rational homology spheres (henceforth called `SFS').
For notations and conventions of the $3$-manifolds and their invariants, see ``Conventions'' below.
Part of the code is based on code written for MAGMA by Karakurt \cite{Karakurt}, which in turn is an implementation of Némethi's algorithm to compute $\hfp$ \cite{Nemethi}. 

The package contains the class \texttt{soapy.SFS}, as well as its subclasses \texttt{soapy.Lens}, \texttt{soapy.Prism} and \texttt{soapy.Brieskorn}.

\section*{\texttt{soapy.SFS}}

An object representing the SFS $Y(e;a_{1}/b_{1},\dots,a_{n}/b_{n})$ (where, $a_{i},b_{i}\neq 0$ for $i=1,\dots,n$.) is created by passing the tuple \texttt{(e;a\_1/b\_1,\dots,a\_n/b\_n)} to the constructor of \texttt{soapy.SFS}.

Alternatively, this can be done by passing the function \texttt{soapy.SFS.from\_plumbing} the weights of a plumbing description of $Y(e;a_{1}/b_{1},\dots,a_{n}/b_{n})$, as specified by a star-shaped tree with integer weights (see ``Example'' below).

If \texttt{Y} is a \texttt{soapy.SFS} object, calling \texttt{-Y} corresponds to orientation-reversal.
Two objects are equal in this class if they represent orientation-preservingly homeomorphic $3$-manifolds.
An object is `less than or equal to' another, if the pair passes the correction term obstruction to there existing a ribbon rational homology cobordism from the first object to the other (see ``Obstruction'' below).
An object is `less than' another if it is less than, but not equal, to the other.

The class contains the objects \texttt{soapy.three\_sphere} and \texttt{soapy.poincare\_sphere}, representing $S^{3}$ and the Poincaré homology sphere, respectively.

\bigskip

\textbf{List of attributes:}

\renewcommand{\arraystretch}{1.5}%
\begin{tabularx}{\linewidth}{X X}
\hline
\texttt{params} & List of integer coefficients representing the SFS specified. These do not necessarily coincide with the input parameters, but rather are normalized in such a way that the corresponding integer surgery diagram is definite.\\
\hline
\texttt{central\_weight} & The weight of the central vertex of the normalized surgery description of the SFS specified.\\
\hline
\texttt{branch\_weights} & Tuple containing the rational surgery coefficients of the exceptional fibers of the SFS specified.\\
\hline
\end{tabularx}

\newpage

\textbf{List of methods:}

\renewcommand{\arraystretch}{1.5}%
\begin{tabularx}{\linewidth}{X X}\\
\hline
\texttt{fractional\_branch\_weights()} & Returns the fractional parts of the framings on the exceptional fibers.\\
\hline
\texttt{euler\_number()} & Returns the orbifold Euler number of the SFS specified.\\
\hline
\texttt{number\_of\_exceptional\_fibers()} & Returns the number of exceptional fibers of the SFS specified.\\
\hline
\texttt{to\_plumbing()} & Returns the definite plumbing (equivalently: an integral surgery description) corresponding to the SFS specified.\\
\hline
\texttt{seifert\_invariants()} & Returns the Seifert invariants of the SFS specified.\\
\hline
\texttt{linking\_matrix()} & Returns the linking matrix of the SFS specified.\\
\hline
\texttt{first\_homology()} & Returns the first homology of the SFS specified.\\
\hline
\texttt{order\_of\_first\_homology()} & Returns the order of the first homology of the SFS specified.\\
\hline
\texttt{spinc\_to\_HF()} & Computes $\hfp$ of the SFS specified in each spin$^c$-structure. The $\Z[U]$-module-structure of $\hfp$ is encoded as a dictionary of the format \{`order of $\Z[U]$-module-summand' : `list of bottommost gradings of all $\Z[U]$-module-summands of that order'\}.\\
\hline
\texttt{print\_HF()} & Prints $\hfp$ of the SFS specified in a more legible manner.\\
\hline
\texttt{correction\_terms()} & Returns a list of the corrections terms of the SFS specified.\\
\hline
\texttt{is\_lspace()} & Checks whether or not the SFS specified is a Heegaard Floer L-space.\\
\hline
\texttt{casson\_walker()} & Computes the Casson-Walker invariant of the SFS specified.\\
\hline
\texttt{is\_lens\_space()} & Checks whether or not the SFS specified is homeomorphic to a lens space.\\
\hline
\texttt{to\_lens\_space()} & Transforms the SFS specified into the corresponding lens space.\\
\hline
\texttt{is\_prism\_mfld()} & Checks whether or not the SFS specified is homeomorphic to a prism manifold.\\
\hline
\texttt{to\_prism\_mfld()} & Transforms the SFS specified into the corresponding prism manifold.\\
\hline
\end{tabularx}

\bigskip

\textbf{List of functions:}

\renewcommand{\arraystretch}{1.5}%
\begin{tabularx}{\linewidth}{X X}
\hline
\texttt{number\_to\_neg\_cont\_frac} & This computes the coefficients of the (-)-continued fraction expansion of a rational number $p/q$, for $p$, $q$ coprime.\\
\hline
\texttt{number\_to\_pos\_cont\_frac} & This computes the coefficients of the (+)-continued fraction expansion of a rational number $p/q$, for $p$, $q$ coprime.\\
\hline
\texttt{number\_from\_neg\_cont\_frac} & This computes (-)-continued fraction of an arbitrary list of coefficients.\\
\hline
\texttt{number\_from\_pos\_cont\_frac} & This computes (+)-continued fraction of an arbitrary list of coefficients.\\
\hline
\end{tabularx}

\newpage

\section*{\texttt{soapy.Lens}}

This is a subclass of \texttt{soapy.SFS} representing lens spaces.
In particular, any object belonging to this subclass inherits all methods defined for objects belonging to \texttt{soapy.SFS}.

An object representing the lens space $L(p,q)$ is created by passing the parameters \texttt{p} and \texttt{q} to the constructor of \texttt{soapy.Lens}.
Alternatively, this can be done by passing the function \texttt{soapy.Lens.from\_linear\_lattice} the weights of a linear lattice giving a plumbing description of $L(p,q)$.

\bigskip

\textbf{List of attributes:}

\renewcommand{\arraystretch}{1.5}%
\begin{tabularx}{\linewidth}{X X}
\hline
\texttt{p} & The first parameter of the lens space specified, normalized to be greater than zero.\\
\hline
\texttt{q} & The second parameter of the lens space specified, normalized so that $p> q > 0$.\\
\hline
\end{tabularx}

\bigskip

\textbf{List of methods:}

\renewcommand{\arraystretch}{1.5}%
\begin{tabularx}{\linewidth}{X X}\\
\hline
\texttt{to\_SFS()} & Transforms the lens space specified into a SFS.\\
\hline
\texttt{to\_linear\_lattice(epsilon=-1)} & Returns the weights of the linear lattice bounded by the lens space specified. By default, the negative definite linear lattice is returned, unless epsilon is set to 1.\\
\hline
\end{tabularx}

\newpage

\section*{\texttt{soapy.Prism}}

This is a subclass of \texttt{soapy.SFS} representing prism manifolds.
In particular, any object belonging to this subclass inherits all methods defined for objects belonging to \texttt{soapy.SFS}.

An object representing the prism manifold $P(p,q)$ is created by passing the parameters \texttt{p} and \texttt{q} to the constructor of \texttt{soapy.Prism}.

\bigskip

\textbf{List of attributes:}

\renewcommand{\arraystretch}{1.5}%
\begin{tabularx}{\linewidth}{X X}
\hline
\texttt{p} & The first parameter of the prism manifold specified, normalized to be greater than 0.\\
\hline
\texttt{q} & The second parameter of the prism manifold specified.\\
\hline
\end{tabularx}

\textbf{List of methods:}

\renewcommand{\arraystretch}{1.5}%
\begin{tabularx}{\linewidth}{X X}
\hline
\texttt{to\_SFS()} & Transforms the prism manifold specified into a SFS. \phantom{aaaaaaaaaaaaaaaaaaaa}\\
\hline
\end{tabularx}

\section*{\texttt{soapy.Brieskorn}}

This is a subclass of \texttt{soapy.SFS} representing Brieskorn homology spheres.
In particular, any object belonging to this subclass inherits all methods defined for objects belonging to \texttt{soapy.SFS}.

An object representing the Brieskorn homology sphere $\Sigma(a_{1},\dots,a_{n})$ is created by passing the parameters \texttt{a\_1} through \texttt{a\_n} to the constructor of \texttt{soapy.Brieskorn}.
If all entries of the tuple are negative, an object representing $-\Sigma(a_{1},\dots,a_{n})$ is created.

\bigskip

\textbf{List of attributes:}

\renewcommand{\arraystretch}{1.5}%
\begin{tabularx}{\linewidth}{X X}
\hline
\texttt{coeffs} & The coefficients of the Brieskorn homology sphere specified. \phantom{aaaa aaaa aaaa}\\
\hline
\end{tabularx}

\bigskip

\textbf{List of methods:}

\renewcommand{\arraystretch}{1.5}%
\begin{tabularx}{\linewidth}{X X}
\hline
\texttt{to\_SFS()} & Transforms the Brieskorn homology sphere specified into a SFS. \phantom{aaaaaaaaaaaaaaaaaaaa}\\
\hline
\end{tabularx}

\subsection*{Conventions}
\begin{itemize}
\item The Seifert fibered space $Y(e;a_{1}/b_{1},\dots,a_{n}/b_{n})$ is the $3$-manifold with surgery diagram given in Figure \ref{diag}.
\begin{figure}[h]
\def\svgwidth{200pt}
\input{diag.pdf_tex}
\caption{}
\label{diag}
\end{figure}
\item The orbifold Euler number of $Y(e;a_{1}/b_{1},\dots,a_{n}/b_{n})$ is defined to be \[e_{\text{orb}}(Y):=e - \sum_{i=1}^{n}\frac{b_{i}}{a_{i}}.\]
\item The lens space $L(p,q)$ is the oriented $3$-manifold obtained by performing $-(p/q)$-framed Dehn surgery along the unknot in $S^{3}$.
Defined for any pair $p,q$ of non-zero coprime integers. 
\item The prism manifold $P(p,q)$ is defined to be $Y(-1;-2,-2,-p/q)$.
Defined for any pair $p,q$ of non-zero coprime integers.
Contrary to some conventions, $p=\pm1$ is allowed; note, however, that $P(1,n)$ is homeomorphic to the lens space $L(4n,2n-1)$.
\item Brieskorn spheres are oriented such that $\Sigma(2,3,5)$ is the boundary of the negative definite plumbing along the $E_{8}$ Dynkin diagram (where each vertex has weight $-2$).
$\Sigma(a_{1},\dots,a_{n})$ is defined for any sequence of positive, pairwise coprime integers.
\item The Casson-Walker invariant is defined as in \cite{Rustamov}, i.e. \[\lambda(Y)=\frac{1}{|H_{1}(Y;\mathbb{Z})|}\sum_{\mathfrak{s}\in\mathrm{Spin}^{c}(Y)}\left(\chi(\mathrm{HF}^{+}_{\mathrm{red}}(Y))-\frac{d(Y,\mathfrak{s})}{2}\right).\]
\end{itemize}

\subsection*{Obstruction}
Given a pair of SFS's $Y_1$ and $Y_2$, the package implements the following obstruction to there being a ribbon rational homology cobordism from $Y_1$ to $Y_2$.
If $Y_1$ admits a ribbon rational homology cobordism to $Y_2$, then
\begin{itemize}
\item $|H_{1}(Y_{2};\mathbb{Z})|=u^{2}\cdot|H_{1}(Y_{1};\mathbb{Z})|$, for some $u\in\mathbb{Z}$; and
\item each correction term of $Y_{1}$ appears $u$ times among the correction terms of $Y_{2}$ (counted with multiplicity)
\end{itemize}

\emph{Example:} Let $Y_{1}=L(2,1)$ and $Y_{2}=L(8,5)$.
The lists of correction terms of $Y_{1}$ and $Y_{2}$ are $[-1/4,\,1/4]$ and $[-3/8,\,-3/8,\,-1/4,\,-1/4,\,1/4,\,1/4,\,5/8,\,5/8]$, respectively.
Since \[|H_{1}(Y_{2};\mathbb{Z})|=8=2^{2}\cdot 2=2^{2}\cdot|H_{1}(Y_{1};\mathbb{Z})|,\] and because each correction term of $Y_{1}$ appears $2$ times among those of $Y_{2}$, the pair passes the obstruction.

Note that, while this implements just an obstruction, pairs of lens spaces (and connected sums thereof) that admit a rational ribbon cobordism from one to the other have been classified in \cite{Huber}

\subsection*{Example}
Calling \texttt{Y = soapy.SFS(2,2,1,3,2,11,9)} creates an object representing $Y(2;2,3/2,11/9)$. Starting from a plumbing diagram, this object can be created by calling \texttt{soapy.SFS.from\_plumbing(2,[2],[2,2],[2,2,2,2,3])} (generally, the lists of weights along the branches must be specified starting at the central vertex).
The coefficients of this plumbing can be found by calling \texttt{Y.to\_plumbing()}.
One can verify that $Y$ is homeomorphic to $-\Sigma(2,3,11)$ e.g. by calling \texttt{Y == -soapy.Brieskorn(2,3,11)} or \texttt{Y == soapy.Brieskorn(-2,-3,-11)}.

Calling \texttt{Y.first\_homology()} returns an empty tuple, thus verifying that $H_{1}(Y;\Z)$ has no non-trivial summands, and hence that $Y$ is an integral homology sphere.

Calling \texttt{Y.print\_HF()} (or, equivalently, \texttt{sp.Brieskorn(2,3,11).print\_HF()}) prints 
\begin{quote}
\texttt{HF\^{}+(Y(2; 2, 3/2, 11/9)):\\
spin\^{}c-structure: (0, 0, 0, 0)\\
HF\^{}+ = T\_(-2) + Z(1)\_(-2)}
\end{quote}
This is to be interpreted as saying that $\hfp(Y)$ in its unique $\text{spin}^{c}$-structure consists of a tower whose bottommost grading is $-2$, together with a $\Z[U]$-torsion summand of order $1$ whose bottommost grading is $-2$ as well.
In other words, \[\hfp(-\Sigma(2,3,11),\mathfrak{s})\cong\mathcal{T}_{(-2)}^{+}\oplus\Z_{(-2)}.\]
Accordingly, calling \texttt{Y.correction\_terms()} and \texttt{Y.is\_lspace} returns \texttt{(-2,)} and \texttt{False}, respectively.

%%%%%%%%%%%%%%%%%%%%%%%%%%%%%%%%%%%%%%%%%%%%%%%%%%%%%%%%%%%%%%%%%%%%%%%%%%%%%%%%%%%%

\bibliographystyle{alpha}
\bibliography{biblio}

\end{document}

%%%%%%%%%%%%%%%%%%%%%%%%%%%%%%%%%%%%%%%%%%%%%%%%%%%%%%%%%%%%%%%%%%%%%%%%%%%%%%%%%%%%