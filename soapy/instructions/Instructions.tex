%------------------------------------------------------------------------------
% Beginning of journal.tex
%------------------------------------------------------------------------------
%
% AMS-LaTeX version 2 sample file for journals, based on amsart.cls.
%
%        ***     DO NOT USE THIS FILE AS A STARTER.      ***
%        ***  USE THE JOURNAL-SPECIFIC *.TEMPLATE FILE.  ***
%
% Replace amsart by the documentclass for the target journal, e.g., tran-l.
%
\documentclass[12pt]{amsart}

%     If your article includes graphics, uncomment this command.
\usepackage{graphicx}
\usepackage{cleveref}
\usepackage{color}
\usepackage{units}
\usepackage{enumitem}

\newtheorem{theorem}{Theorem}[section]
\newtheorem{lemma}[theorem]{Lemma}

\theoremstyle{definition}
\newtheorem{definition}{Definition}
\newtheorem{example}[theorem]{Example}
\newtheorem{xca}[theorem]{Exercise}

\theoremstyle{remark}
\newtheorem{remark}[theorem]{Remark}
\newtheorem*{remark*}{Remark}
\newtheorem*{example*}{Example}
\newtheorem*{examples*}{Examples}

\numberwithin{equation}{section}

%    Absolute value notation
\newcommand{\abs}[1]{\lvert#1\rvert}

%    Blank box placeholder for figures (to avoid requiring any
%    particular graphics capabilities for printing this document).
\newcommand{\blankbox}[2]{%
  \parbox{\columnwidth}{\centering
%    Set fboxsep to 0 so that the actual size of the box will match the
%    given measurements more closely.
    \setlength{\fboxsep}{0pt}%
    \fbox{\raisebox{0pt}[#2]{\hspace{#1}}}%
  }%
}

\setlength{\textwidth}{\paperwidth}
\addtolength{\textwidth}{-4cm}
\calclayout

\newcommand{\hfp}{\mathrm{HF}^{+}}

\begin{document}

\title{Instructions}
\vspace*{-3.5cm}
\maketitle
\thispagestyle{empty}

This packet contains four programs: \texttt{CF}, \texttt{HF\_SFS}, \texttt{CW} and \texttt{RIB\_OBS}.
For each of these, the underlying code (\texttt{*.py}) as well as a more user-friendly version with a minimalistic interface is included; the latter can be run by running \texttt{python3 *\_user\_dialogue.py} in a terminal.
These programs require Python 3.8, and the Python library \texttt{SymPy} should be installed (also see Pipfiles).
All files should be located in the same directory.

\subsection*{Instructions for \texttt{CF}}

The functions \texttt{number\_to\_neg\_cont\_frac} and \texttt{number\_to\_pos\_cont\_frac} compute (the coefficients of) the negative and positive continued fraction expansions, respectively, of a rational number.
The coefficients that these functions return are at least 2 in absolute value and all of the same sign as the given rational number.
The functions \texttt{neg\_cont\_frac\_to\_number} and \texttt{pos\_cont\_frac\_to\_number} execute the reverse process, i.e. compute a rational number given (the coefficients of) its negative or positive continued fraction expansion, respectively.

\begin{examples*}
\leavevmode
\begin{itemize}[leftmargin=*]
\item \texttt{number\_to\_neg\_cont\_frac(18,5) = [4,3,2]} and \texttt{neg\_cont\_frac\_to\_number(4,3,2) = 18/5},\\because \[\dfrac{18}{5}=4-\dfrac{1}{3-\dfrac{1}{2}};\]
\item \texttt{number\_to\_pos\_cont\_frac(30,7) = [4,3,2]} and \texttt{pos\_cont\_frac\_to\_number(4,3,2) = 30/7},\\because \[\dfrac{30}{7}=4+\dfrac{1}{3+\dfrac{1}{2}};\]
\item \texttt{number\_to\_neg\_cont\_frac(-18,5)= number\_to\_neg\_cont\_frac(18,-5) = [-4,-3,-2]},\\because \[-\dfrac{18}{5}=-4-\dfrac{1}{-3-\dfrac{1}{-2}}.\]
\end{itemize}
\end{examples*}

\subsection*{Instructions for \texttt{HF\_SFS}}

This program computes $\hfp$ of Seifert fibered spaces (SFS) that are rational homology spheres.
It is largely based on a corresponding code written for MAGMA by Karakurt \cite{Karakurt}, which in turn is an implementation of Némethi's algorithm to compute $\hfp$ \cite{Nemethi}.
The main function is \texttt{spinc\_to\_HF}, which outputs $\hfp(Y)$ of a SFS $Y$, encoded as a dictionary of the format 
\begin{center}
\texttt{\string{spinc-structure : $\mathbb{Z}[U]$-module-structure of $\hfp$\string}}
\end{center}
(see example below on how to interpret this dictionary).
The function \texttt{spinc\_to\_HF} takes as input a list of integers which should be of the format \texttt{[e, a\_1, b\_1, ..., a\_n, b\_n]}, where $e$ and the $a_i$ and the $b_i$ are integers specifying $Y$ as in Figure \ref{diag} and such that the corresponding plumbing (obtained by expanding the $a_i/b_i$ into continued fractions) is definite.
\begin{figure}[ht]
\def\svgwidth{200pt}
\input{diag.pdf_tex}
\caption{}
\label{diag}
\end{figure}

To compute $\hfp$ of a lens space $L(p,q)$, a prism manifold $P(p,q)$ or a Brieskorn sphere $\Sigma(a_{1},\dots,a_{n})$, the input list can be computed using the functions \texttt{lens(p,q)}, \texttt{prism(p,q)} and \texttt{brieskorn(a\_1, ..., a\_n)}, respectively (see below for orientation conventions).
Calling the function \texttt{print\_HF} will return $\hfp(Y)$ in a more legible manner.
Finally, building on \texttt{spinc\_to\_HF}, the function \texttt{correction\_terms} returns the list of correction terms of $Y$, and \texttt{is\_lspace} checks whether $Y$ is an L-space.

\begin{remark*}
The input list for $-\Sigma(a_{1},\dots,a_{n})$ is computed using \texttt{brieskorn(-a\_1, ..., -a\_n)}. The functions \texttt{lens(p,q)} and \texttt{prism(p,q)} take any pair of non-zero integers as arguments, so that e.g. \texttt{lens(8,5)}, \texttt{lens(-8,-5)}, \texttt{lens(8,13)} and \texttt{lens(8,-3)} each return the input list corresponding to the lens space $L(8,5)$.
\end{remark*}

\begin{example*}
The Brieskorn sphere $\Sigma(2,3,7)$ bounds a (negative) definite plumbing along the star-shaped tree on four vertices with three branches, where the central vertex has weight $-1$ and the leaves have weights $-2$, $-3$ and $-7$.
The corresponding input list is \texttt{[-1,-2,1,-3,1,-7,1]}.
Given this input, \texttt{spinc\_to\_HF} returns the dictionary \texttt{\string{(0, 0, 0, 0) : \string{0:[0], 1:[-1]\string}\string}}.
This is to be interpreted as saying that in the (unique) $\mathrm{Spin}^{c}$-structure labelled \texttt{(0,0,0,0)}, $\hfp(\Sigma(2,3,7))$ consists of an infinite tower whose bottommost element is in grading $0$, and a tower of length 1 (i.e. a $\mathbb{Z}$-summand) whose bottommost element is in grading $-1$.
That is, \[\hfp(\Sigma(2,3,7)) = \mathcal{T}^{+}_{(0)}\oplus\mathbb{Z}_{(-1)}.\]
The same output can be obtained by calling \texttt{spinc\_to\_HF(brieskorn(2,3,7))}, and also be displayed in the above, more legible manner by calling \texttt{print\_HF([-1,-2,1,-3,1,-7,1])}.

The above output shows that the (single) correction term of $Y$ equals 0, which could have been computed by calling e.g. \texttt{correction\_terms(brieskorn(2,3,7))}.

The above output also shows that $\Sigma(2,3,7)$ is not an L-space; this could have been verified by calling either \texttt{is\_lspace(brieskorn(2,3,7))} or \texttt{is\_lspace([-1,-2,1,-3,1,-7,1])}, which returns \texttt{False}.
\end{example*}

\subsection*{Instructions for \texttt{CW}}

This program computes the Casson-Walker invariant of Seifert fibered spaces that are rational homology spheres.
Generally, $\lambda(Y)$ is computed by calling the function \texttt{casson\_walker}, whose input should be a list of integers of the same format as is used for \texttt{spinc\_to\_HF} (see above).
In the case where $Y$ is a lens space or a prism manifold, the functions \texttt{casson\_walker\_lens} and \texttt{casson\_walker\_prism} can be called instead.

\begin{remark*}
While the Casson-Walker invariant can be extracted from $\hfp(Y)$ (see ``Conventions'' below), closed formulas for $\lambda(Y)$ are used in \texttt{CW} in order to increase speed.
These underlying formulas can be found in \cite[Section 5.3]{Nemethi-Nicolaescu} (for the general case and the lens space case) and in \cite[Section 2.1]{Ballinger_3} (for the prism manifold case).
\end{remark*}

\begin{example*}
$\lambda(\Sigma(2,3,7))=-1$; this can be obtained by calling \texttt{casson\_walker(brieskorn(2,3,7))} or \texttt{casson\_walker([-1,-2,1,-3,1,-7,1])}.
\end{example*}

\subsection*{Instructions for \texttt{RIB\_OBS}}

This program can be used to check whether a pair of SFS's $Y_1$ and $Y_2$ passes the following obstruction to there being a ribbon rational homology cobordism from $Y_1$ to $Y_2$.
If $Y_1$ admits a ribbon rational homology cobordism to $Y_2$, then
\begin{itemize}
\item $|H_{1}(Y_{2};\mathbb{Z})|=u^{2}\cdot|H_{1}(Y_{1};\mathbb{Z})|$, for some $u\in\mathbb{Z}$; and
\item each correction term of $Y_{1}$ appears $u$ times among the correction terms of $Y_{2}$ (counted with multiplicity)
\end{itemize}
The function \texttt{ribbon\_obstruction\_corr} checks whether a pair of rational homology spheres passes this obstruction if the list of their correction terms is known, and takes these two lists as its input.

The function \texttt{ribbon\_obstruction\_sfs} checks whether a pair of SFS's passes this obstruction given definite plumbings filling the two SFS's (where each plumbing is entered as a list of integers as in \texttt{spinc\_to\_HF}).
Again, there are shortcut functions \texttt{ribbon\_obstruction\_lens} and \texttt{ribbon\_obstruction\_prism} for the case where both SFS's are lens spaces or prism manifolds, respectively.

\begin{example*}
Let $Y_{1}=L(2,1)$ and $Y_{2}=L(8,5)$.
The lists of correction terms of $Y_{1}$ and $Y_{2}$ are $[1/4,\,-1/4]$ and $[1/4,\,-1/4,\,-3/8,\,5/8,\,-3/8,\,-1/4,\,5/8,\,1/4]$, respectively.
Since \[|H_{1}(Y_{2};\mathbb{Z})|=8=2^{2}\cdot 2=2^{2}\cdot|H_{1}(Y_{1};\mathbb{Z})|,\] and because each correction term of $Y_{1}$ appears $2$ times among those of $Y_{2}$, the pair passes the obstruction.
This can be verified by calling \texttt{ribbon\_obstruction\_corr([1/4, -1/4],[1/4, -1/4, -3/8, 5/8, -3/8, -1/4, 5/8, 1/4])}, which returns \texttt{True}.

Since $Y_{1}$ and $Y_{2}$ bound plumbings whose input lists are \texttt{[-2]} and \texttt{[-3,-2,1,-2,1]}, respectively, the same output can be obtained by calling \texttt{ribbon\_obstruction\_sfs([-2],[-3,-2,1,-2,1])}.

Alternatively, since both $Y_{1}$ and $Y_{2}$ are lens spaces, this output can be obtained by calling \texttt{ribbon\_obstruction\_lens(2,1,8,5)}.
\end{example*}

\subsection*{Conventions}
\begin{itemize}
\item The lens space $L(p,q)$ is the oriented $3$-manifold obtained by performing $-(p/q)$-framed Dehn surgery along the unknot in $S^{3}$.
\item The prism manifold $P(p,q)$ is the $3$-manifold obtained by setting $n=3$, $a_{1}/b_{1}=a_{2}/b_{2}=-2/1$ and $a_{3}/b_{3}=-p/q$ in Figure \ref{diag}.
\item Brieskorn spheres are oriented such that $\Sigma(2,3,5)$ is the boundary of the negative definite plumbing along the $E_{8}$ Dynkin diagram (where each vertex has weight $-2$).
\item The Casson-Walker invariant is defined as in \cite{Rustamov}, i.e. \[\lambda(Y)=\frac{1}{|H_{1}(Y;\mathbb{Z})|}\sum_{\mathfrak{s}\in\mathrm{Spin}^{c}(Y)}\left(\chi(\mathrm{HF}^{+}_{\mathrm{red}}(Y))-\frac{d(Y,\mathfrak{s})}{2}\right).\]
\end{itemize}

%%%%%%%%%%%%%%%%%%%%%%%%%%%%%%%%%%%%%%%%%%%%%%%%%%%%%%%%%%%%%%%%%%%%%%%%%%%%%%%%%%%%

\bibliographystyle{alpha}
\bibliography{biblio}

\end{document}

%%%%%%%%%%%%%%%%%%%%%%%%%%%%%%%%%%%%%%%%%%%%%%%%%%%%%%%%%%%%%%%%%%%%%%%%%%%%%%%%%%%%